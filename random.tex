\documentclass[12pt,a4paper]{article}
\usepackage{amsfonts, amssymb, amsmath}
\usepackage{fullpage}
\usepackage{parskip} % skip a line instead of indenting
% \usepackage{graphicx} % for inserting images
\usepackage{amsthm}
\usepackage{xcolor}
\usepackage{tikz} % for plot

\newtheorem*{rem}{Remark}

\title{Database Systems}
\author{R4 Cheng}
\date{\today}

\newcommand{\Remark}[1]{
  \begin{rem}
    \color{cyan}
    #1
  \end{rem}
}

\begin{document}
\maketitle

Best practice:

\begin{enumerate}
    \item Always prepare your own key
\end{enumerate}

\Remark{Say what you want instead of how to do}

Use IN/Not in to check if a value is in a set

\section*{Datalog}

Each query is a rule.

E.g. For all values of Part, Subpart, and Qty,

\textbf{if} there is a tuple \(\langle \text{Part, Subpart, Qty} \rangle\) in Assembly,

\textbf{then} there must be a tuple \(\langle \text{Part, Subpart} \rangle\) in Components.

\begin{verbatim}
    Components(Part, Subpart) :- Assembly(Part, Subpart, Qty).
\end{verbatim}

E.g. For all values of Part, Part2, Subpart, and Qty,

\textbf{if} there is a tuple \(\langle \text{Part, Part2, Qty} \rangle\) in Assembly, 
\textbf{and} a tuple \(\langle \text{Part2, Subpart} \rangle\) in Components,

\textbf{then} there must be a tuple \textless Part, Subpart\textgreater in Components.

\begin{verbatim}
    Components(Part, Subpart) :- Assembly(Part, Part2, Qty), 
                               Components(Part2, Subpart).
\end{verbatim}

\Remark{Each application of a Datalog rule can be understood in terms of relational algebra}

\subsection*{Unsafe rules}

\begin{verbatim}
    (Unsafe) V(x, y, z) :- Actor(x, y, 1998), z > 200
\end{verbatim}

This is unsafe because \textbf{z} is not bound to any relation, meaning it can take on infinitely many values.

\begin{verbatim}
    (Unsafe) W(x,y,z) :- Actor(x,y,z), not Plays(t,x)
\end{verbatim}

This is unsafe because The variable \textbf{t} appears only in the negated literal not Plays(t, x) and does not appear in any positive literal in the body

\Remark{Every variable should appear in at least one positive body atom}

\section*{Relational Algebra}

\begin{itemize}
    \item Defines a set of basic operations on relations
    \item Each operation returns a relation
    \item \textbf{Result} of an operation can be the input of another operation
\end{itemize}

Basic operations:

\begin{itemize}
    \item Selection ($\sigma$): Selects a subset of rows from relation.
    \item Projection ($\pi$): Deletes attributes that are not in the projection list and deletes duplicate rows.
    \item Union ($\cup$): Tuples in relation 1 and in relation 2.
    \item Set-difference ($-$): Tuples in relation 1 but not in relation 2.
    \item Cross product ($\times$): Allows us to combine two relations. it returns all possible pairs of tuples from the two relations.
\end{itemize}

Join is a combination of selection and cross product.

\[ R \bowtie S = \sigma_{condition} (R \times S) \]

\section*{Relational Calculus}

\begin{itemize}
    \item First-order logic
    \item Tuple relational calculus (TRC)
    \item Domain relational calculus (DRC)
\end{itemize}

Each relational predicate P is:

\begin{itemize}
    \item Atom (Actor(x, y, z))
    \item P $\land$ P (conjunction)
    \item P $\lor$ P (disjunction)
    \item P $\Rightarrow$ P (implication)
    \item $\lnot$ P (negation)
    \item $\forall$ x P (for all x P holds)
    \item $\exists$ x P (for an x P holds)
\end{itemize}

\subsection*{Examples}

Exists a schema:

$Movie(\underline{mid},title,year,total-gross)$ \\
$Actor(\underline{aid},name,b-year)$ \\
$Plays(\underline{mid},\underline{aid})$

Q: Actor who played only in movies produced in 1990

$Result(x) = \forall{y}.Play(y, x) \Rightarrow \exists{z}\exists{t}.Movie(y, z, 1990,t)$

\subsection*{Tuple Relational Calculus}

Form: $\{T \mid p(T)\}$

The result of this query is the set of all tuples $t$ for which the formula $p(T)$ evaluates to true with $T = t$.

\subsection*{Domain Relational Calculus}

Form: $\{<x_1,x_2,\dots,x_n> \mid p(<x_1,x_2,\dots,x_n>)\}$, 
where each $x_i$ is either a domain variable or a constant
and $p(<x_1, x_2,\dots,x_n>)$ denotes a \textbf{DRC formula}.

\section*{Index}

An \textbf{index} is a data structure that organizes data records on disk to optimize certain kinds of retrieval operations.

We use the term \textbf{data entry} to refer to the records stored in an index file.

There are three main alternatives for what to store as a data entry in an index:

\begin{enumerate}
    \item A data entry $k*$ is an actual data record (with searh key value $k$).
    \item A data entry $<k, rid>$ pair, where $rid$ is the record id of a data record with search key value $k$.
    \item A data entry $<k, red-list>$ pair, where $red-list$ is a list of record ids of data records with search key value $k$. 
\end{enumerate}

\begin{enumerate}
    \item Hash-based Indexing
    \item Tree-based Indexing
\end{enumerate}

clustered index:

unclustered index:

sop:

\subsection*{JOIN Algorithms}



\subsection*{TO TA}

\begin{itemize}
    \item why in 2 pass merge sort the number of M unit increases?
    \item book 9.7 why to point free space there?
\end{itemize}

\subsection*{TODO}

\begin{itemize}
    \item fully understand pages, frames, and buffer pool
\end{itemize}

\end{document}